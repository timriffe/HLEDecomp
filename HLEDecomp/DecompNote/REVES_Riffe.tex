\documentclass[20pt,usenames,dvipsnames]{beamer}

\usepackage{tikz}
\usepackage[normalem]{ulem}
\geometry{paperwidth=10in, paperheight=7.5in}
\usepackage{animate}
\usepackage{xcolor,colortbl}
\usepackage{booktabs}
\usepackage[utf8]{inputenc}
\usepackage{pgfplots}
\usetikzlibrary{arrows,calc,positioning,shapes.geometric}
\pgfplotsset{compat=1.15}
%\usepackage[mpidr]{./mpidr/beamerthemeMPIDR}
%\usefonttheme{serif}
%\newcolumntype{C}[1]{>{\centering\let\newline\\\arraybackslash\hspace{0pt}}m{#1}}
%\newcommand*{\QEDA}{\hfill\ensuremath{\blacksquare}}
%% Declaring title and author
%	the institute's logo
%\renewcommand{\mylogo}{\includegraphics[width=4.7in]{mpidr_logo_colour_en}}
\usepackage{color}
\definecolor{mygray}{rgb}{0.8,0.8,0.8}
\definecolor{yellow}{rgb}{1,1,0}

\defbeamertemplate{description item}{align left}{\insertdescriptionitem\hfill}
%%	should be the very last package to be loaded
\usepackage{hyperref}
\newcommand{\white}[1]{\textcolor{white}{#1}}
\newcommand{\blue}[1]{\textcolor{blue}{#1}}
%%%%%%%%%%%%%%%%%%%%%%%%%%%%%%%%%%
%%	Beginning of the document		%%
%%%%%%%%%%%%%%%%%%%%%%%%%%%%%%%%%%
\begin{document}

%%	titlepage - fixed frame:
%%	========================

% \begin{frame}
% 	\titlepage
% \end{frame}
\begin{frame}[plain]
	%\titlepage
	\vspace{-.5cm}
 \centerline{\includegraphics[scale=1]{Figs/ehu_logo_strip.png}}

	
	\huge
	\vspace{1em}
	\textbf{\white{Consider parameterizing in terms of conditional probabilities when} decomposing \white{discrete time} multistate models}\\
	\vspace{1em}
	\large 
	Tim Riffe \\
	\vspace{1em}
	28 May, 2021\\
	REVES annual meeting
\end{frame}

% -------------------------------------------------------------
\begin{frame}[plain]
	%\titlepage
	\vspace{-.5cm}
 \centerline{\includegraphics[scale=1]{Figs/ehu_logo_strip.png}}

	
	\huge
	\vspace{1em}
	\textbf{\white{Consider parameterizing in terms of conditional probabilities when} decomposing discrete time multistate models}\\
	\vspace{1em}
	\large 
	Tim Riffe \\
	\vspace{1em}
	28 May, 2021\\
	REVES annual meeting
\end{frame}


% -------------------------------------------------------------

\begin{frame}[plain]
	%\titlepage
	\vspace{-.5cm}
 \centerline{\includegraphics[scale=1]{Figs/ehu_logo_strip.png}}

	
	\huge
	\vspace{1em}
	
	\textbf{Consider parameterizing in terms of \blue{conditional probabilities} when decomposing discrete time multistate models}\\
	\vspace{1em}
	\large 
	Tim Riffe \\
	\vspace{1em}
	28 May, 2021\\
	REVES annual meeting
\end{frame}


%-------------------

\begin{frame}[plain]{A typical multistate model}

\small
\begin{center}
\begin{tikzpicture}[scale=0.7, ->,>=stealth',shorten >=1pt,auto,node distance=3.5cm,
  thick,main node/.style={circle,draw},square/.style={regular polygon,regular polygon sides=4}]

  \node[main node] (1) at (0,0) [square,draw]{Healthy};
  \node[main node] (2) at (8,-5) [square,draw] {Disabled};
  \node[main node] (3) at (3,-12) [square,draw] {Dead};  

  \path[every node/.style={font=\sffamily\small}]
    
    (1) edge node [left] {} (3)
   	(1) edge [bend right] node [left] {} (2)
    
    (2) edge node [right] {} (3)
    (2) edge [bend right] node [left] {} (1)
    
    (1) edge [loop left] node {} (1)
    (2) edge [loop right] node {} (2)
    (3) edge [loop below] node {} (3)
 
    ;
\end{tikzpicture}
\end{center}
\end{frame}


\begin{frame}[plain]{A typical multistate model}

\small
\begin{center}
\begin{tikzpicture}[scale=0.7, ->,>=stealth',shorten >=1pt,auto,node distance=3.5cm,
  thick,main node/.style={circle,draw},square/.style={regular polygon,regular polygon sides=4}]

  \node[main node] (1) at (0,0) [square,draw]{Healthy};
  \node[main node] (2) at (8,-5) [square,draw] {Disabled};
  \node[main node] (3) at (3,-12) [square,draw] {Dead};  

  \path[every node/.style={font=\sffamily\small}]
    
    (1) edge node [left] {} (3)
   	(1) edge [bend right, red] node [left] {} (2)
    
    (2) edge node [right] {} (3)
    (2) edge [bend right, red] node [left] {} (1)
    
    (1) edge [loop left, red] node {} (1)
    (2) edge [loop right, red] node {} (2)
    (3) edge [loop below] node {} (3)
 
    ;
\end{tikzpicture}
\end{center}
\end{frame}

% -----------------------------------

\begin{frame}[plain]
\Large
$$\xi = f(\color{red}\theta \color{black})$$
\pause
where $\xi$ can be any synthetic quantity calculated with $\color{red}\theta$.

- often $\xi$ is an expectancy
\end{frame}

% -----------------------------------

\begin{frame}[plain]{setup}
\Large
$$ \Delta \xi = \xi^2 - \xi^1$$
\pause
$$ = f(\color{red}{\theta^2} \color{black}) - f(\color{red}{\theta^1} \color{black})$$
\pause
$$ \Delta\xi = \sum \mathbf{c}_i$$
\pause
$$ \mathbf{c} = \mathcal{D}(f, \color{red}{\theta^2}\color{black},\color{red}{\theta^1}\color{black})$$
\end{frame}

% -----------------------------------

\begin{frame}[plain]{Decomposition, $\mathcal{D}()$}
\begin{itemize}
\item LTRE (Caswell 1989)
\item Stepwise (Andreev et al 2002)
\item Pseudo continuous (Horiuchi et al 2008)
\end{itemize}

\end{frame}

% -----------------------------------

\begin{frame}[plain]

\Huge
\centering
Let's talk about \scalebox{2}{\color{red}$\theta$}

\end{frame}

% -----------------------------------

\begin{frame}[plain]

\Large
Pick two colors to make \scalebox{1.5}{\color{red}$\theta$}

\begin{center}
\includegraphics[]{Figs/Transitions.pdf}
\end{center}
\end{frame}

% -----------------------------------

\begin{frame}[plain]
\definecolor{gray}{RGB}{100,100,100}
\begin{minipage}{.48\linewidth}
\resizebox{\textwidth}{!}{%
\begin{tikzpicture}[ ->,>=stealth',shorten >=1pt,auto,node distance=1.5cm,
  thick,main node/.style={circle,draw},square/.style={regular polygon,regular polygon sides=4}]

  \node[main node] (1) at (0,0) [square,draw]{Healthy};
  \node[main node] (2) at (8,-5) [square,draw] {Disabled};
  \node[main node] (3) at (3,-12) [square,draw] {Dead};  

  \path[every node/.style={font=\sffamily\small}]
    
    (1) edge [gray] node [left] {} (3)
   	(1) edge [bend right, red] node [left] {} (2)
    
    (2) edge [gray] node [right] {} (3)
    (2) edge [bend right, red] node [left] {} (1)
    
    (1) edge [loop left, red] node {} (1)
    (2) edge [loop right, red] node {} (2)
    (3) edge [loop below, gray] node {} (3)
 
    ;
\end{tikzpicture}
}
\end{minipage}
\begin{minipage}{.48\linewidth}
\resizebox{\textwidth}{!}{%
\begin{tikzpicture}[->,>=stealth',shorten >=1pt,auto,node distance=3.5cm,
  thick,main node/.style={circle,draw},square/.style={regular polygon,regular polygon sides=4}]

  \node[main node] (1) at (0,0) [square,draw]{Healthy};
  \node[main node] (2) at (8,-5) [square,draw] {Disabled};
  \node[main node] (3) at (3,-12) [square,draw] {Dead};  

  \path[every node/.style={font=\sffamily\small}]
    
    (1) edge [blue] node [left] {} (3)
   	(1) edge [bend right, blue] node [left] {} (2)
    
    (2) edge [blue] node [right] {} (3)
    (2) edge [bend right, blue] node [left] {} (1)
    
    (1) edge [loop left, gray] node {} (1)
    (2) edge [loop right, gray] node {} (2)
    (3) edge [loop below, gray] node {} (3)
 
    ;
\end{tikzpicture}
}
\end{minipage}
\end{frame}

% -----------------------------------

\begin{frame}[plain]
\Large
$$\xi = \color{red}{f(\theta)} \color{black} = \color{blue}{f(\theta)}$$
\pause
$$ \Delta \xi = \color{red}{\xi^2 - \xi^1} \color{black} = \color{blue}{\xi^2 - \xi^1}$$
\pause
$$ \mathcal{D}(\color{red}{f}, \color{red}{\theta^2}\color{black},\color{red}{\theta^1}\color{black}) \ne \mathcal{D}(\color{blue}{f}, \color{blue}{\theta^2}\color{black},\color{blue}{\theta^1}\color{black})$$

$$ \sum \color{red}{\mathbf{c}}^{\color{black}i}\color{black} =\sum \color{blue}{\mathbf{c}}^{\color{black}i} $$
but
$$
\color{red}{\mathbf{c}}^{\color{black}i} \ne \color{blue}{\mathbf{c}}^{\color{black}i}$$
\end{frame}

% -----------------------------------

\begin{frame}[plain]{Example}
\Large
DFLE increased from 30.75 in 2006 to 32.33 in 2014.
That's $\Delta \xi = $1.58 years

\bigskip
\bigskip
\small (HRS, age 50 women with secondary education)

\end{frame}

\begin{frame}[plain]{Example}

\Large
Same result, $\xi$  whether we omit:

\begin{itemize}
\item self-transitions
\item mortality transitions
\item health transitions
\end{itemize}

But very different stories if we decompose:

\normalsize
\begin{table}[ht]
\centering
\begin{tabular}{r|rrrrrr}
  \hline
omits & Stay healthy & Get disabled & Die healthy & Recover & Stay disab. & Die disabled \\ 
  \hline
self &  & -0.01 & 1.32 & -0.28 &  & 0.54 \\ 
  mort & 1.28 & 0.04 &  & -1.86 & 2.13 &  \\ 
  health & 0.21 &  & 1.10 &  & -0.41 & 0.67 \\ 
   \hline
\end{tabular}
\end{table}
\end{frame}

\begin{frame}[plain]{"Thank you" intermission}
\begin{center}
\includegraphics[]{Figs/thanks.png}
\end{center}
\end{frame}

\begin{frame}[plain]
\Large
We would like a solution that gives consistent interpretable results
\pause

\bigskip

\Huge
Solution

\bigskip

\Large
\pause
Make $\theta$ consist in conditional probabilities
\end{frame}

\begin{frame}[plain]
\Large
For standard calcs we use (two of) 

$$\left[p^{stay}, p^{switch}, p^{die}\right]$$

\pause
Transform this into two multiplicative probabilities 

$$\left[p^{stay} | survive, p^{survive}\right]$$
\end{frame}

\begin{frame}[plain]{Complementarity}
\begin{table}[ht]
\centering
\begin{tabular}{cccc}
  \hline
 DF mort & Dis. mort & DF\rightarrow Dis & Dis\rightarrow Df \\ \hline
1.29   &   0.58   &     0.02   &       -0.31
\end{tabular}
\end{table}
\pause

Transitions can be framed in terms of mortality or survival, in terms of staying in the state of transfering out of it. Results \color{blue}\emph{identical}

\pause
\Huge
\color{blue} Really, IDENTICAL

\pause
\color{black}
Thanks
\end{frame}

\end{document}





