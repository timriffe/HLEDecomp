\documentclass[20pt,usenames,dvipsnames]{beamer}

\usepackage{tikz}
\usepackage[normalem]{ulem}
\geometry{paperwidth=10in, paperheight=7.5in}
\usepackage{animate}
\usepackage{xcolor,colortbl}
\usepackage{booktabs}
\usepackage[utf8]{inputenc}
\usepackage{pgfplots}
\usetikzlibrary{arrows,calc,positioning,shapes.geometric}
\pgfplotsset{compat=1.15}
%\usepackage[mpidr]{./mpidr/beamerthemeMPIDR}
%\usefonttheme{serif}
%\newcolumntype{C}[1]{>{\centering\let\newline\\\arraybackslash\hspace{0pt}}m{#1}}
%\newcommand*{\QEDA}{\hfill\ensuremath{\blacksquare}}
%% Declaring title and author
%	the institute's logo
%\renewcommand{\mylogo}{\includegraphics[width=4.7in]{mpidr_logo_colour_en}}
\usepackage{color}
\definecolor{mygray}{rgb}{0.8,0.8,0.8}
\definecolor{yellow}{rgb}{1,1,0}

\defbeamertemplate{description item}{align left}{\insertdescriptionitem\hfill}
%%	should be the very last package to be loaded
\usepackage{hyperref}
\newcommand{\white}[1]{\textcolor{white}{#1}}
\newcommand{\blue}[1]{\textcolor{blue}{#1}}
\newcommand{\presentpic}{\includegraphics[scale = .5]{Figs/present.png}}
%%%%%%%%%%%%%%%%%%%%%%%%%%%%%%%%%%
%%	Beginning of the document		%%
%%%%%%%%%%%%%%%%%%%%%%%%%%%%%%%%%%
\begin{document}

%%	titlepage - fixed frame:
%%	========================

% \begin{frame}
% 	\titlepage
% \end{frame}
\begin{frame}[plain]
	%\titlepage
	\vspace{.5cm}
 \centerline{\includegraphics[scale=1]{Figs/ehu_logo_strip2.png}}

	\huge
	\vspace{5mm}
	\textbf{\white{Consider parameterizing in terms of conditional probabilities when} decomposing \white{discrete time} multistate models}\\
	\vspace{1em}
	\large 
	Tim Riffe \\
	\vspace{5mm}
	11 Nov, 2021\\
	CED Seminari
\end{frame}

% -------------------------------------------------------------
%\begin{frame}[plain]
%	%\titlepage
%	\vspace{.5cm}
% \centerline{\includegraphics[scale=1]{Figs/ehu_logo_strip.png}}
%
%	\huge
%	\vspace{1em}
%	\textbf{\white{Consider parameterizing in terms of conditional probabilities when} decomposing discrete time %multistate models}\\
%	\vspace{1em}
%	\large 
%	Tim Riffe \\
%	\vspace{1em}
%	28 May, 2021\\
%	REVES annual meeting
%\end{frame}


% -------------------------------------------------------------

\begin{frame}[plain]
	%\titlepage
	\vspace{.5cm}
 \centerline{\includegraphics[scale=1]{Figs/ehu_logo_strip2.png}}

	\huge
	\vspace{5mm}
	
	\textbf{Consider parameterizing in terms of \blue{conditional probabilities} when decomposing discrete time multistate models}\\
	\vspace{1em}
	\large 
	Tim Riffe \\
	\vspace{5mm}
	11 Nov, 2021\\
	CED Seminari
\end{frame}


%-------------------
\begin{frame}[plain]{What is a multistate model?}
\url{https://temery86.github.io/FullHistory/}
\vspace{1em}

\includegraphics[scale=.5]{Figs/multistate_animation.png}
\end{frame}


\begin{frame}[plain]{A typical multistate model}

\small
\begin{center}
\begin{tikzpicture}[scale=0.7, ->,>=stealth',shorten >=1pt,auto,node distance=3.5cm,
  thick,main node/.style={circle,draw},square/.style={regular polygon,regular polygon sides=4}]

  \node[main node] (1) at (0,0) [square,draw]{Healthy};
  \node[main node] (2) at (8,-5) [square,draw] {Disabled};
  \node[main node] (3) at (3,-12) [square,draw] {Dead};  

  \path[every node/.style={font=\sffamily\small}]
    
    (1) edge node [left] {} (3)
   	(1) edge [bend right] node [left] {} (2)
    
    (2) edge node [right] {} (3)
    (2) edge [bend right] node [left] {} (1)

    (1) edge [loop left] node {} (1)
    (2) edge [loop right] node {} (2)
    (3) edge [loop below] node {} (3)
 
    ;
\end{tikzpicture}
\end{center}
\end{frame}


\begin{frame}[plain]{A typical multistate model}

\small
\begin{center}
\begin{tikzpicture}[scale=0.7, ->,>=stealth',shorten >=1pt,auto,node distance=3.5cm,
  thick,main node/.style={circle,draw},square/.style={regular polygon,regular polygon sides=4}]

  \node[main node] (1) at (0,0) [square,draw]{Healthy};
  \node[main node] (2) at (8,-5) [square,draw] {Disabled};
  \node[main node] (3) at (3,-12) [square,draw] {Dead};  

  \path[every node/.style={font=\sffamily\small}]
    
    (1) edge node [left] {} (3)
   	(1) edge [bend right, red] node [left] {} (2)
    
    (2) edge node [right] {} (3)
    (2) edge [bend right, red] node [left] {} (1)
    
    (1) edge [loop left, red] node {} (1)
    (2) edge [loop right, red] node {} (2)
    (3) edge [loop below] node {} (3)
 
    ;
\end{tikzpicture}
\end{center}
\end{frame}

% -----------------------------------

\begin{frame}[plain]
\Large
$$\presentpic = f(\color{red}\theta \color{black})$$
\vspace{1em}
 \presentpic \enspace is any synthetic index calculated from $\color{red}\theta$
 
\end{frame}

% -----------------------------------

\begin{frame}[plain]{Decomposition asbtract}
\Large
$$ \Delta\presentpic = \presentpic^2 - \presentpic^1$$

$$ = f(\color{red}{\theta^2} \color{black}) - f(\color{red}{\theta^1} \color{black})$$

$$ \Delta\presentpic = \sum \mathbf{c}_i$$

$$ \mathbf{c} = \mathcal{D}(f, \color{red}{\theta^2}\color{black},\color{red}{\theta^1}\color{black})$$
\end{frame}

% -----------------------------------

\begin{frame}[plain]{Decomposition, \scalebox{1.5}{$\mathcal{D}()$}}
\begin{itemize}
\item difference-scaled partial derivatves a.k.a \emph{LTRE} (Caswell 1989)
\item Stepwise parameter swapping (Andreev et al 2002)
\item Pseudo continuous (Horiuchi et al 2008)
\end{itemize}

\end{frame}

% -----------------------------------

\begin{frame}[plain]

\Huge
\centering
Let's talk about \scalebox{2}{\color{red}$\theta$}

\end{frame}

% -----------------------------------

\begin{frame}[plain]

\Large
Pick two colors to make \scalebox{1.5}{\color{red}$\theta$}

\begin{center}
\includegraphics[]{Figs/Transitions.pdf}
\end{center}
\end{frame}

% -----------------------------------

\begin{frame}[plain]
\definecolor{gray}{RGB}{100,100,100}
\begin{minipage}{.48\linewidth}
\resizebox{\textwidth}{!}{%
\begin{tikzpicture}[ ->,>=stealth',shorten >=1pt,auto,node distance=1.5cm,
  thick,main node/.style={circle,draw},square/.style={regular polygon,regular polygon sides=4}]

  \node[main node] (1) at (0,0) [square,draw]{Healthy};
  \node[main node] (2) at (8,-5) [square,draw] {Disabled};
  \node[main node] (3) at (3,-12) [square,draw] {Dead};  

  \path[every node/.style={font=\sffamily\small}]
    
    (1) edge [gray] node [left] {} (3)
   	(1) edge [bend right, red] node [left] {} (2)
    
    (2) edge [gray] node [right] {} (3)
    (2) edge [bend right, red] node [left] {} (1)
    
    (1) edge [loop left, red] node {} (1)
    (2) edge [loop right, red] node {} (2)
    (3) edge [loop below, gray] node {} (3)
 
    ;
\end{tikzpicture}
}
\end{minipage}
\begin{minipage}{.48\linewidth}
\resizebox{\textwidth}{!}{%
\begin{tikzpicture}[->,>=stealth',shorten >=1pt,auto,node distance=3.5cm,
  thick,main node/.style={circle,draw},square/.style={regular polygon,regular polygon sides=4}]

  \node[main node] (1) at (0,0) [square,draw]{Healthy};
  \node[main node] (2) at (8,-5) [square,draw] {Disabled};
  \node[main node] (3) at (3,-12) [square,draw] {Dead};  

  \path[every node/.style={font=\sffamily\small}]
    
    (1) edge [blue] node [left] {} (3)
   	(1) edge [bend right, blue] node [left] {} (2)
    
    (2) edge [blue] node [right] {} (3)
    (2) edge [bend right, blue] node [left] {} (1)
    
    (1) edge [loop left, gray] node {} (1)
    (2) edge [loop right, gray] node {} (2)
    (3) edge [loop below, gray] node {} (3)
 
    ;
\end{tikzpicture}
}
\end{minipage}
\end{frame}

% -----------------------------------
%
%\begin{frame}[plain]
%\Large
%$$\presentpic = \color{red}{f(\theta)} \color{black} = \color{blue}{f^\prime(\theta)}$$
%
%$$ \Delta \presentpic = \color{red}{\presentpic^2 - \presentpic^1} \color{black} = \color{blue}{\presentpic^2 - %\presentpic^1}$$
%
%$$ \mathcal{D}(\color{red}{f}, \color{red}{\theta^2}\color{black},\color{red}{\theta^1}\color{black}) \ne %\mathcal{D}(\color{blue}{f^\prime}, \color{blue}{\theta^2}\color{black},\color{blue}{\theta^1}\color{black})$$
%
%$$ \sum \color{red}{\mathbf{c}}^{\color{black}i}\color{black} =\sum \color{blue}{\mathbf{c}}^{\color{black}i} %$$
%
%$$
%\color{red}{\mathbf{c}}^{\color{black}i} \ne \color{blue}{\mathbf{c}}^{\color{black}i}$$
%\end{frame}

% -----------------------------------

\begin{frame}[plain]{Example}
\Large
DFLE increased from 30.75 in 2006 to 32.33 in 2014.
That's $\Delta \presentpic = $1.58 years

\bigskip
\bigskip
\small (HRS, age 50 women with secondary education)

\end{frame}

\begin{frame}[plain]{Example}

\Large
Same result, \presentpic\enspace  whether we omit:

\begin{itemize}
\item self-transitions
\item mortality transitions
\item health transitions
\end{itemize}

But \color{red}very different stories \color{black} if we decompose:

\normalsize
\begin{table}[ht]
\centering
\begin{tabular}{r|rrrrrr}
  \hline
$\theta$ omits & DF\rightarrow DF & DF\rightarrow Dis & DF mort & Dis\rightarrow DF & Dis\rightarrow Dis & Dis mort \\ 
  \hline
\color{blue}self &  & \color{blue}-0.01 & \color{blue}1.32 & \color{blue}-0.28 &  & \color{blue}0.54 \\ 
  \color{red} mort & \color{red}1.28 & \color{red}0.04 &  & \color{red}-1.86 & \color{red}2.13 &  \\ 
  health & 0.21 &  & 1.10 &  & -0.41 & 0.67 \\ 
   \hline
\end{tabular}
\end{table}
\end{frame}

\begin{frame}[plain]{"Thank you" intermission}
\begin{center}
\includegraphics[]{Figs/thanks.png}
\end{center}
\end{frame}

\begin{frame}[plain]
\Large
We would like a solution that gives consistent interpretable results


\bigskip

\Huge
\begin{center}
Solution
\end{center}
\bigskip

\Large

Make $\theta$ consist in conditional probabilities
\end{frame}

\begin{frame}[plain]
\Large
For standard calcs compose $\theta$ from (two of) 

$$\left[p^{stay}, p^{switch}, p^{die}\right]$$

Transform this into two multiplicative probabilities 

$$\left[p^{stay} | survive, p^{survive}\right]$$

\end{frame}

\begin{frame}[plain]{Complementarity (or \emph{Symmetry}?)}
\begin{table}[ht]
\centering
\begin{tabular}{cccc}
  \hline
 DF mort & Dis. mort & DF\rightarrow Dis & Dis\rightarrow Df \\ \hline
1.29   &   0.58   &     0.02   &       -0.31
\end{tabular}
\end{table}

Transitions can be framed in terms of mortality or survival, in terms of staying in the state of transfering out of it. Results \color{blue}\emph{identical}

\pause
\Huge
\color{blue} Really, IDENTICAL

\pause
\color{black}
Thanks
\end{frame}

\end{document}





