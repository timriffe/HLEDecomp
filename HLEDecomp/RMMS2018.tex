%\begin{filecontents*}{example.eps}
%!PS-Adobe-3.0 EPSF-3.0
%%BoundingBox: 19 19 221 221
%%CreationDate: Mon Sep 29 1997
%%Creator: programmed by hand (JK)
%%EndComments

%\end{filecontents*}
%
%\documentclass{svjour3}                     % onecolumn (standard format)
%\documentclass[smallcondensed]{svjour3}     % onecolumn (ditto)
\documentclass[12pt,oneside,a4paper]{article}  

\usepackage{apacite}
\usepackage{appendix}
\usepackage{amsmath}
\usepackage{amsthm}

\usepackage{amssymb} % for approx greater than
\usepackage{caption}
\usepackage{placeins} % for \FloatBarrier
\usepackage{graphicx}
%\usepackage{subcaption}
\usepackage{longtable}
\usepackage{setspace}
\usepackage{booktabs}
\usepackage{tabularx}
\usepackage{xcolor,colortbl}
\usepackage{chngpage}
\usepackage{natbib}
\bibpunct{(}{)}{,}{a}{}{;} 
\usepackage{url}
\usepackage{nth}
\usepackage{authblk}
\usepackage[most]{tcolorbox}
\usepackage[normalem]{ulem}
\usepackage{amsfonts}

% columns for longtable
%\usepackage{arydshln} % Dashed lines in matrices

\usepackage[margin=1in]{geometry}
%\doublespacing % for review

% line numbers to make review easier
%\usepackage{lineno}
%\linenumbers

%\usepackage{soul}% for \st{}

%%%%%%%%%%%%%%%%%%%%%%%%%%%%%%%%%%%%%%%%%%%%%%%%%%%%%%%%%%%%%%%%%%%%%%%%%%%%%%
% for section 4 math environments
%\theoremstyle{definition}
%\newtheorem{definition}{Definition}[section]
%\newtheorem{theorem}{Theorem}[section]
%\newtheorem{proposition}{Proposition}[section]
%\newtheorem{corollary}{Corollary}[proposition]
%\newtheorem{remark}{Remark}[section]
%
%%%%%%%%%%%%%%%%%%%%%%%%%%%%%%%%%%%%%%%%%%%%%%%%%%%%%%%%%%%%%%%%%%%%%%%%%%%%%%
%\begin{filecontents*}{example.eps}
%!PS-Adobe-3.0 EPSF-3.0
%%BoundingBox: 19 19 221 221
%%CreationDate: Mon Sep 29 1997
%%Creator: programmed by hand (JK)
%%EndComments
%gsave
%newpath
%  20 20 moveto
%  20 220 lineto
%  220 220 lineto
%  220 20 lineto
%closepath
%2 setlinewidth
%gsave
%  .4 setgray fill
%grestore
%stroke
%grestore
%\end{filecontents*}
%\RequirePackage{fix-cm}

\newcommand\ackn[1]{%
  \begingroup
  \renewcommand\thefootnote{}\footnote{#1}%
  \addtocounter{footnote}{-1}%
  \endgroup
}

% Affiliations in small font size
%\renewcommand\Affilfont{\small}
\newcommand{\absdiv}[1]{%
  \par\addvspace{.5\baselineskip}% adjust to suit
  \noindent\textbf{#1}\quad\ignorespaces
}

%\defcitealias{HMD}{HMD 2016}

% junk for longtable caption
%\AtBeginEnvironment{longtable}{\linespread{1}\selectfont}
%\setlength{\LTcapwidth}{\linewidth}

% sort van Raalte properly
% #1: sorting key, #2: prefix for citation, #3: prefix for bibliography
%\DeclareRobustCommand{\VAN}[3]{#2} % set up for citation
%\newcommand{\tc}{\quad\quad\text{,}}
%\newcommand{\tp}{\quad\quad\text{.}}
%%%%%%%%%%%%%%%%%%%%%%%%%%%%%%%
\begin{document}


\title{Healthy lives: Delayed onset, improved recovery, or mortality
change?}

%\author{Tim Riffe \and Neil Mehta \and Daniel Schneider \and Mikko Myrskyl\"a}
\author[1]{Tim Riffe\thanks{riffe@demogr.mpg.de}}
\author[2]{Neil Mehta}
\author[1]{Daniel Schneider}
\author[1,3]{Mikko Myrskyl\"a}

\affil[1]{Max Planck Institute for Demographic Research}
\affil[2]{University of Michigan, Ann Arbor}
\affil[3]{University of Helsinki}

%\authorrunning{Short form of author list} % if too long for running head

%\institute{   Tim Riffe \at
%              Max Planck Institute for Demographic Research.
%              Konrad-Zuse-Str. 1. 18057 Rostock, Germany\\
%              \email{riffe@demogr.mpg.de}\\
%              Tel.:  +49 176 232 858 45\\
%              Fax: +49 381 2081 - 280
%\and 
% Neil Mehta \at
%              Department of Health Management and Policy. 
%School of Public Health.
%M3531 SPH II
%Ann Arbor, MI 48109-2029\\
%              \email{nkmehta@umich.edu }
%\and
%Daniel Schneider \at
%              Max Planck Institute for Demographic Research.
%              Konrad-Zuse-Str. 1. 18057 Rostock, Germany\\
%              \email{schneider@demogr.mpg.de}      
%\and
%Mikko Myrskyl\"a \at
%              Max Planck Institute for Demographic Research.
%              Konrad-Zuse-Str. 1. 18057 Rostock, Germany\\
%              \email{myrskyla@demogr.mpg.de}   
%              }
%              
%

\maketitle

\vspace{-2em}
\begin{abstract}
\absdiv{Background} 
Healthy life expectancy at older ages in the United States has steadily
increased in recent decades. We do not know whether changes in
disease onset, recovery, or mortality drive this trend.
\absdiv{Objective}
We aim to determine how much of the change in healthy and unhealthy
life expectancy between 1995 and 2015 is due to changes in onset, recovery, and
mortality.
\absdiv{Data and Methods}
We use the US Health and Retirement Study
to estimate transition rates between health and mild and severe disability
states, as well as state-specific death rates, for the years 1995,
2004, and 2014. We calculate remaining healthy, disabled, and total life
expectancy at age 50 using incidence-based Markov matrix models. We decompose the difference
between time points and population strata into 9 separate age-specific components for onset,
recovery, and mortality using pseudo-continuous decomposition.
\absdiv{Results}
We describe preliminary results for males, all education groups combined.
Perhaps counter to intuition, most change in healthy life expectancy is due to
mortality and not to onset of or recovery from disability. Most of the two-year
increase in healthy life expectancy since 1995 is due to decreased mortality of
healthy people, whereas delayed onset and slowed recovery from
disability offset each other. Expected years in mild disability increased by
about 4 months over the two decades, mostly due to improved mortality of both
healthy and mildly disabled people. Delayed onset of mild disability almost
equally offset the effects of improved mortality among the mildly disabled.
Expected years in severe disability increased by about half a year, also mostly
due to improved mortality in all health states. 
\absdiv{Conclusions}
Healthy life expectancy at age 50 increased relatively faster than disabled life
expectancy, both driven by mortality improvements. Years spent in disability
have been pushed into higher ages, indicating a slight delay of onset.
\end{abstract}



% bibliography
%\bibliographystyle{spbasic}
%\bibliography{references}  
\end{document}
